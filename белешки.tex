% !TeX encoding = UTF-8
% !TeX spellcheck = mk_MK-Macedonian
\documentclass[12pt,a4paper]{article}
\usepackage[english,macedonian]{babel}
\usepackage[LCY,OT1]{fontenc}
\usepackage{markdown}
\usepackage{amsmath}
\usepackage{mathtools}

\usepackage[a4paper,left=3cm,right=2cm,top=2.5cm,bottom=2.5cm]{geometry}

\begin{document}
	
	\title{Линеарна алгебра и примени}
	\author{}
	\date{}
	\maketitle
	
	\section{07.10.2022}
	\subsection{Линеарни равенки}
		Равенка која може да се изрази како: $a_{1}x_{1}+a_{2}x_{2}+...+a_{a}x_{n}=b$ каде $x_i$ се \textbf{променливи или слободни членови}, а $a_i$ се \textbf{коефициенти} се нарекува линеарна равенка.
		\\ 
		\textit{\textbf{$a_i$ може да бидат реални или комплексни и често се знае нивниот тип претходно.}}
		\\ \\
		$a_i$ исто така се нарекуваат и параметри, можат да бидат и изрази под услов да не содржат променливи (слободни членови) во себе. За да се добие корисна равенка, потребно е сите коефициенти да не се нули.
		
	
	\subsection{Систем од линеарни равенки}
		Систем од линеарни равенки (или линеарен систем) претставува колекција на една или повеќе линеарни равенки кои вклучуваат исти променливи.
		Пример:
		\begin{align}
			\begin{cases}
				2x_1 - x_2 + 1.5x_3 = 8 \\
				x_1 - 4x_3 = -7
			\end{cases}
		\end{align}
		
		Решение на системот е \textbf{листа} $(s_1, s_2, ..., s_n)$ од броеви кои кога ќе се заменат со сите x-овци во секоја од равенките левата и десната страна ќе бидат исти.
		
		Решението на системот (1) е листата (5, 6.3, 3) бидејќи кога вредностите од листата ќе се постават наместо x-овците, левите и десните страни и кај двете равенки стануваат исти.
		\\ \\ 
		Подредената n-торка која ја задоволува секоја равенка во системот се нарекува \textbf{решение (партикуларно решение)}.
		Множество решенија на системот се нарекува \textbf{општо решение}.
		Да се реши системот значи да се најде општо решение на системот, односно множество од сите n-торки кои ја задоволуваат равенката.
		\\ \\
		Два линеарни система се \textbf{еквивалентни} ако го содржат истото множество на решенија (општото решение).
		\\ \\
		Бидејќи систем линеарни равенки со две непознати содржи информации за две линии, пронаоѓањето на множеството решенија \textbf{(општо решение)} на дадениот линеарен систем се сведува на пронаоѓање на сите пресеци на двете линии.
		\begin{itemize}
			\item Доколку линиите се сечат \textbf{точно} во една точка, тогаш системот има точно едно решение.
			\item Доколку линиите се \textbf{паралелни}, тогаш системот нема решенија.
			\item Доколку линиите \textbf{лежат една на друга}, тогаш системот има бесконечно многу решенија.
		\end{itemize}
		
		За еден систем се вели дека е \textbf{конзистентен} доколку има или едно или бесконечно многу решенија, аналогно на тоа, \textbf{неконзистентен} систем е оној кој нема решенија.
		
	\subsection{Изразување на линеарна равенки со матрица}
	Сите потребни информации за линеарен систем може да се претстават користејќи матрица.
	Пример, системот:
	
	\begin{align}
		\begin{cases}
			x_1 - 2x_2 + x_3 = 0 \\
			2x_2 - 8x_3 = 8 \\
			5x_1 - 5x_3 = 10	
		\end{cases}
	\end{align}
	\\
	може да се претстави како \textbf{матрица на коефициенти}:
	\begin{align}
		\begin{bmatrix*}[r]
			1 & -2 & 1\\
			0 & 2 & -8\\
			5 & 0 & -5
		\end{bmatrix*}
	\end{align}

	Дополнително може да се претстави како \textbf{проширена матрица}:
	\begin{align}
		\begin{bmatrix*}[r]
			1 & -2 & 1 & \vline & 0\\
			0 & 2 & -8 & \vline & -8\\
			5 & 0 & -5 & \vline & -5
		\end{bmatrix*}
	\end{align}

	Проширената матрица се состои од матрица на коефициенти со дополнителна колона (Матрица, вектор, од резултати) која ја претставува константата од десна страна на секоја од равенките.
	
	Големината на матрицата ни кажува колку редови и колони има. Проширената матрица има 3 реда и 4 колони, односно претставува 3 x 4 матрица. 
	\\
	\textit{\textbf{Бројот на редици се појавува пред бројот на колони.}}
	
	\subsection{Решавање на линеарен систем}
	Едноставната стратегија за решавање на линеарен систем е да се замени системот со еквивалентен систем (со исто множество на решенија) кој е поедноставен за решавање. 
	Ако на еден систем од линеарни равенки се применат следните трансформации се добива систем еквивалентен со провобитниот:
	\begin{itemize}
		\item Замена на местата на две равенки од системот.
		\item Множење на една равенка со ненулти број.
		\item Множење на една равенка од системот со ненулти број и додавање на друга равенка од системот.
	\end{itemize}
	\subsubsection{Пример: решавање на линеарен систем преку елиминирање на членови}
	\begin{align}
		\begin{cases}
			x_1 - 2x_2 + x_3 = 0 \\
			2x_2 - 8x_3 = 8 \\
			5x_1 - 5x_3 = 10
		\end{cases}
		\text{Претставен како матрица на коефициенти:}
		\begin{bmatrix*}[r]
			1 & -2 & 1 & \vline & 0 \\
			0 & 2 & -8 & \vline & 8 \\
			5 & 0 & -5 & \vline & 10
		\end{bmatrix*}
	\end{align}

	Најпрактично е кај линеарната равенка со која ќе работиме (одземаме или додаваме на друга равенка) да се доведе коефициентот пред променливата, онаа која ни е од корист, на единица со цел да поедноставиме во пресметките, односно целата равенка да ја поделиме со вредноста на константата пред променливата. 
	
	Нека остане $x_1$ присутен само во првата равенка, а елиминиран во сите останати.
	\\ \\
	\textbf{Постапка на елиминирање на $x_1$:}
	
	Бидејќи $x_1$ во системот се појавува во две линеарни равенки од кои во една се појавува со коефициент 1 паметниот пристап ќе биде да ја искористиме наведената равенка (со коефициент 1) како основа за множење со чија помош ќе го елиминираме $x_1$ од другите равенки.
	\begin{align}
		\frac{\begin{array}{c}
			\text{[Равенка 3]} \\
			-(5 * \text{[Равенка 1]})
		\end{array}}{
			\text{[Нова 3 равенка]}
	}
	\end{align}
	\begin{align}
		\frac{\begin{array}{c}
			5x_1 - 5x_3 = 10 \\
			-(5x_1 - 10x_2 + 5x_3 = 0)
		\end{array}}{
			10x_2 - 10x_3 = 10
	}
	\end{align}
	Линеарниот систем по извршената операција:
		\begin{align}
		\begin{cases}
			x_1 - 2x_2 + x_3 = 0 \\
			2x_2 - 8x_3 = 8 \\
			10x_2 - 10x_3 = 10
		\end{cases}
		\begin{bmatrix*}[r]
			1 & -2 & 1 & \vline & 0 \\
			0 & 2 & -8 & \vline & 8 \\
			0 & 10 & -10 & \vline & 10
		\end{bmatrix*}
	\end{align}
	Се цел да си ги поедноставиме пресметките константата пред $x_2$ во вториот ред ќе ја доведеме до единица. Тоа ќе го постигнеме со множење на целата равенка со \textbf{1/[константа пред променлива]}. Во овој пример равенката врз која ќе ја примениме оваа операција ќе биде втората.
	\begin{align}
		\frac{1}{2} \cdot (2x_2 - 8x_3 = 8) \to x_2 - 4x_3 = 4
	\end{align}
	Линеарниот систем по извршената операција:
	\begin{align}
		\begin{cases}
			x_1 - 2x_2 + x_3 = 0 \\
			x_2 - 4x_3 = 4 \\
			10x_2 - 10x_3 = 10
		\end{cases}
		\begin{bmatrix*}[r]
			1 & -2 & 1 & \vline & 0 \\
			0 & 1 & -4 & \vline & 4 \\
			0 & 10 & -10 & \vline & 10
		\end{bmatrix*}
	\end{align}
	\textbf{Постапка на елиминирање на $x_2$}
	
	Ја множиме втората равенка по 10 и резултатот го одземаме од третата равенка со цел да се ослободиме од $x_2$.
		\begin{align}
		\frac{\begin{array}{c}
				\text{[Равенка 3]} \\
				-(10 * \text{[Равенка 2]})
		\end{array}}{
			\text{[Нова 3 равенка]}
		}
	\end{align}
	\begin{align}
		\frac{\begin{array}{c}
				10x_2 - 10x_3 = 10 \\
				+ (- 10x_2 + 40x_3 = -40)
		\end{array}}{
			30x_3 = -30
		}
	\end{align}
	Линеарниот систем по извршената операција:
	\begin{align}
		\begin{cases}
			x_1 - 2x_2 + x_3 = 0 \\
			x_2 - 4x_3 = 4 \\
			30x_3 = 30
		\end{cases}
		\begin{bmatrix*}[r]
			1 & -2 & 1 & \vline & 0 \\
			0 & 1 & -4 & \vline & 4 \\
			0 & 0 & 30 & \vline & -30
		\end{bmatrix*}
	\end{align}
	Преку префрлување на коефициентот пред $x_3$ (во трета равенка) од друга страна на еднакво добиваме дека $x_3 = -1$. Со земана на $x_3$ со нејзината вредност во вториот ред (втората равенка) добиваме дека $x_2 = 0$. Преку замена на $x_2$ и $x_3$ со нивните вредности во првиот ред (првата равенка) добиваме дека $x_1=1$.
	
	Добиеното решение ни кажува дека единственото решение на овој систем е \textbf{(1, 0, -1)}. Преку проверка се добива дека резултатот е точен.
\end{document}
